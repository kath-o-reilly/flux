\documentclass[11pt,oneside,letterpaper]{article}

% graphicx package, useful for including eps and pdf graphics
\usepackage{graphicx}
\DeclareGraphicsExtensions{.pdf,.png,.jpg}

% basic packages
\usepackage{color} 
\usepackage{parskip}
\usepackage{float}

% text layout
\usepackage{geometry}
\geometry{textwidth=15cm} % 15.25cm for single-space, 16.25cm for double-space
\geometry{textheight=22cm} % 22cm for single-space, 22.5cm for double-space

% helps to keep figures from being orphaned on a page by themselves
\renewcommand{\topfraction}{0.85}
\renewcommand{\textfraction}{0.1}

% bold the 'Figure #' in the caption and separate it with a period
% Captions will be left justified
\usepackage[labelfont=bf,labelsep=period,font=small]{caption}

% review layout with double-spacing
%\usepackage{setspace} 
%\doublespacing
%\captionsetup{labelfont=bf,labelsep=period,font=doublespacing}

% cite package, to clean up citations in the main text. Do not remove.
\usepackage{cite}
%\renewcommand\citeleft{(}
%\renewcommand\citeright{)}
%\renewcommand\citeform[1]{\textsl{#1}}

% Remove brackets from numbering in list of References
\renewcommand\refname{\large References}
\makeatletter
\renewcommand{\@biblabel}[1]{\quad#1.}
\makeatother

\usepackage{authblk}
\renewcommand\Authands{ \& }
\renewcommand\Authfont{\normalsize \bf}
\renewcommand\Affilfont{\small \normalfont}

% comments
\usepackage{ulem}
\definecolor{blue}{rgb}{0.324,0.609,0.708}
\definecolor{purple}{rgb}{0.459,0.109,0.538}
\def\tb#1#2{\sout{#1} \textcolor{purple}{#2}} 
\def\tbc#1{\textcolor{purple}{[#1]}}
\def\ms#1#2{\sout{#1} \textcolor{blue}{#2}} 
\def\msc#1{\textcolor{blue}{[#1]}}

% notation
\usepackage{amsmath}
\usepackage{amssymb}
\newcommand{\normal}{\mathcal{N}}					% Normal distribution
\setlength{\arraycolsep}{2pt}
\newcommand{\twomatrix}[2]{\left( \begin{matrix} #1 \\ #2 \end{matrix} \right)}								% pretty inline matrix 
\newcommand{\fourmatrix}[4]{\left( \begin{matrix} #1 & #2 \\ #3 & #4 \end{matrix} \right)}					% pretty inline matrix 

%%% TITLE %%%
\title{\vspace{1.0cm} \LARGE \bf 
Supporting Text: \\
Revealing the competitive dynamics of influenza viruses through evolutionary cartography
}

\author[1]{Trevor Bedford}
\author[2,3,4]{Marc A. Suchard}
\author[5]{Philippe Lemey}
\author[1]{Gytis Dudas}
\author[6]{Victoria Gregory}
\author[6]{Alan J. Hay}
\author[6]{John W. McCauley}
\author[7]{Colin Russell}
\author[7,8]{Derek Smith}
\author[1,9]{Andrew Rambaut}

\affil[1]{Institute of Evolutionary Biology, University of Edinburgh, Edinburgh, UK}
\affil[2]{Department of Biomathematics, David Geffen School of Medicine at UCLA, University of California, Los Angeles CA, USA}
\affil[3]{Department of Human Genetics, David Geffen School of Medicine at UCLA, University of California, Los Angeles CA, USA}
\affil[4]{Department of Biostatistics, UCLA Fielding School of Public Health, University of California, Los Angeles CA, USA}
\affil[5]{Department of Microbiology and Immunology, Katholieke Universiteit Leuven, Leuven, Belgium}
\affil[6]{Division of Virology, MRC National Institute for Medical Research, Mill Hill, London, UK}
\affil[7]{Department of Zoology, University of Cambridge, Cambridge, UK.}
\affil[8]{Department of Virology, Erasmus Medical Centre, Rotterdam, Netherlands.}
\affil[9]{Fogarty International Center, National Institutes of Health, Bethesda, MD, USA.}

\date{}

\begin{document}

\maketitle

%%% SUPPORTING METHODS %%%
\section*{Supporting Methods}

\subsection*{Genetic, antigenic and surveillance data}

We compiled an antigenic dataset of hemagglutination inhibition (HI) measurements of virus isolates against post-infection ferret sera for influenza A/H3N2 by collecting data from previous publications \cite{Hay01,Smith04,Russell08,Barr10}, NIMR vaccine strain selection reports for 2002 and 2008--2012 \cite{NIMR02,NIMRMarch08,NIMRFeb09,NIMRFeb10,NIMRSep10,NIMRSep11,NIMRFeb12} and the Feb 2011 VRBPAC report \cite{Cox11FDA}.
We queried the Influenza Research Database \cite{IRD} and the EpiFlu Database \cite{GISAID} for HA nucleotide sequences by matching strain names, e.g.\ A/HongKong/1/1968, and only strains for which sequence was present was retained.
If a strain had multiple sequences in the databases we preferentially kept the IRD sequence and preferentially kept the longest sequence in IRD. 
Sequences were aligned using MUSCLE v3.7 under default parameters \cite{MUSCLE}.
This dataset had 2051 influenza isolates (present as either virus or serum in HI comparisons) dating from 1968 to 2011. 
However, the majority of isolates were present from 2002 to 2007. 
Because we are interested in longer-term antigenic evolution, we subsampled the data to have at most 20 virus isolates per year, preferentially keeping those isolates with more antigenic comparisons. 
We then kept only those serum isolates that are relatively informative to the antigenic placement of viruses, dropping serum isolates that are compared to 4 or fewer different virus isolates.
This censoring left 402 virus isolates, 519 serum isolates and 10,059 HI measurements. 

Antigenic data for influenza A/H1N1 was collected from previous publications \cite{Kendal78,Webster79,Nakajima79,Nakajima81,Chakraverty82,Pereira82,Chakraverty86,Cox83,Daniels85,Raymond86,Stevens87,Donatelli93,Hay01,Daum02,McDonald07,Barr10} and NIMR vaccine strain selection reports for 2002--2010 \cite{NIMR02,NIMR03,NIMR04,NIMRFeb05,NIMRSep05,NIMRMarch06,NIMRSep06,NIMRMarch07,NIMRSep07,NIMRMarch08,NIMRSep08,NIMRFeb09,NIMRFeb10}.
The same procedure was followed as was followed for H3N2 to match sequence data and to subsample antigenic comparisons.
This procedure yielded 115 virus isolates, 77 serum isolates and 1882 HI measurements over the course of 1977 to 2009.

Antigenic comparisons for influenza B/Victoria were collated from previous publications \cite{Rota90, Hay01, Muyanga01, Shaw02, Ansaldi04, Puzelli04, Xu04, Barr06, Daum06, Lin07} and vaccine strain selection reports for 2002--2012 \cite{AusWHO06, NIMR02, NIMR03, NIMR04, NIMRFeb05, NIMRSep05, NIMRMarch06, NIMRSep06, NIMRMarch07, NIMRSep07, NIMRMarch08, NIMRFeb09, NIMRSep09, NIMRFeb10, NIMRSep10, NIMRFeb11, NIMRSep11, NIMRFeb12}.
Here, the sequence matching and subsampling procedure yielded 179 virus isolates, 70 serum isolates and 2003 HI measurements over the course of 1986 to 2011.

Antigenic comparisons for influenza B/Yamagata were collected from previous publications \cite{Rota90, Kanegae90, Nakajima92, Nerome98, Hay01, Muyanga01, Nakagawa02, Abed03, Ansaldi03, Ansaldi04, Matsuzaki04, Puzelli04, Shaw02, Xu04, Barr06, Daum06, Lin07} and vaccine strain selection reports for 2002--2012 \cite{AusWHO06, NIMR02, NIMR03, NIMR04, NIMRFeb05, NIMRSep05, NIMRMarch06, NIMRSep06, NIMRMarch07, NIMRSep07, NIMRMarch08, NIMRFeb09, NIMRSep09, NIMRFeb10, NIMRSep10, NIMRFeb11, NIMRSep11, NIMRFeb12}.
For B/Yamagata, the matching and subsampling procedure resulted in 174 virus isolates, 69 serum isolates and 1962 HI measurements over the course of 1987 to 2011.

Surveillance data was obtained from the Centers of Disease Control and Prevention FluView Influenza Reports from the yearly summaries of influenza seasons 1997--1998 to 2010--2011 \cite{CDCReports}.
As an example, one report states ``collaborating laboratories in the United States tested 195,744 respiratory specimens for influenza viruses, 27,682 (14\%) of which were positive. Of these, 18,175 (66\%) were positive for influenza A viruses, and 9,507 (34\%) were positive for influenza B viruses. Of the 18,175 specimens positive for influenza A viruses, 7,631 (42\%) were subtyped; 6,762 (87\%) of these were seasonal influenza A (H1N1) viruses, and 869 (13\%) were influenza A (H3N2) viruses.''
In this case, we estimate the relative proportion of A/H3N2 of the four clades as $0.66 \times 0.13 = 0.09$.
Similar calculations were performed for A/H1N1, B/Vic and B/Yam.

\subsection*{Estimation of diffusion coefficient}

The distance $d$ traveled by two-dimensional diffusion is not a linear function of time interval $t$, and thus the `rate' of diffusion cannot be calculated following $r=d/t$.
Assume we have a one-dimensional diffusion starting from $x_0 = 0$, so that
\begin{equation}
	X_t \sim \normal(0, \sigma^2 t),
\end{equation}
where $\sigma$ represents the diffusion volatility, and in this case, the expected squared displacement follows
\begin{equation}
	\mathrm{E}[X_t^2] = \sigma^2 t.
\end{equation}
In moving to a two-dimensional diffusion
\begin{equation}
	(X_t, Y_t) \sim \normal \left( \twomatrix{0}{0}, \fourmatrix{\sigma^2 t}{0}{0}{\sigma^2 t} \right),
\end{equation}
a factor of 2 is included when calculating expected squared displacement
\begin{equation}
	\mathrm{E}[X_t^2 + Y_t^2] = 2 \sigma^2 t.
\end{equation}
Such diffusions are often parameterized with diffusion coefficient $D$ rather than volatility parameter $\sigma$, in this case $D=\sigma^2/2$ and 
\begin{equation}
	\mathrm{E}[X_t^2 + Y_t^2] = 4 D t.
\end{equation}

Owing to this relationship, Pybus et al.\ \cite{Pybus12} estimate $D$ from phylogenies in which internal nodes have inferred 2D locations via the formula
\begin{equation} \label{pybusest}
	\hat{D} = \frac{1}{n} \sum^n_{i=1} \frac{d_i^2}{4t_i},
\end{equation}
in which $d_i$ represents the displacement of a branch and $t_i$ represents the length of a branch.
However, short branches along the phylogeny may cause $d_i^2/t_i$ to have high variance and lead to uncertainty in the resulting estimates of $D$.
Consequently, we estimate $D$ by looking at the relationship between total squared displacement along branches of a phylogeny and total time along branches of a phylogeny
\begin{eqnarray}
	d^2_\mathrm{total} = \sum^n_{i=1} d^2_i \nonumber \\
	t_\mathrm{total} = \sum^n_{i=1} t_i^2 \nonumber \\
	\hat{D} = \frac{ d^2_\mathrm{total} }{ 4 t_\mathrm{total} }.
\end{eqnarray}
This estimator showed less variance than the estimator in equation \ref{pybusest} when analyzing tree output from the BMDS model and was thus preferred.
When estimating $D$ specific to trunk branches or side branches, we calculate $d^2_\mathrm{total}$ and $t_\mathrm{total}$ only from the subset of trunk branches or side branches as appropriate.

%%% REFERENCES %%%
\bibliographystyle{plos}
\bibliography{flux}

\end{document}