\documentclass[stdletter,letterpaper,addrfromright,orderfromdateto,dateleft,11pt,noaddrto,sigleft]{newlfm}
\topmarginskip{-0.4in}
\bottommarginskip{-1.5in}
\leftmarginsize{1in}
\rightmarginsize{1.25in}
\sigskipbefore{0.2in}
\sigskipafter{0in}
\noLines
\nolines
\noHeadline
\noheadline
\signature{Trevor Bedford}

\namefrom{}
\addrfrom{Inst.\ of Evolutionary Biology \\ University of Edinburgh \\ Ashworth Laboratories \\ King's Buildings \\ Edinburgh, UK}
\phonefrom{(617) 285-2542}
\faxfrom{(734) 763-0544}
\emailfrom{bedfordt@umich.edu}

\greetto{Dear Prof.\ Levin,}
\closeline{Sincerely,}

% comments
\usepackage{color} 
\usepackage{ulem}
\definecolor{purple}{rgb}{0.459,0.109,0.538}
\def\tb#1#2{\sout{#1} \textcolor{purple}{#2}} 
\def\tbc#1{\textcolor{purple}{[#1]}}
 
\begin{document}

\begin{newlfm}

Please find, attached, our manuscript entitled ``Integrating influenza antigenic dynamics with molecular evolution,'' which we would be grateful for your consideration for publication in \textit{Proc Natl Acad Sci USA}.  
We believe that \textit{Proc Natl Acad Sci USA} is an appropriate venue for this work, as it clarifies long-standing questions about the antigenic and evolutionary dynamics of influenza viruses, while simultaneously presenting a major methodological advance in the characterization of antigenic phenotype.

Since the publication of ``Mapping the Antigenic and Genetic Evolution of Influenza Virus'' by Smith et al.\ in 2004, there has been significant interest in assessing patterns of antigenic and genetic evolution in seasonal influenza.
Such cartographic approaches use multidimensional scaling (MDS) to position viruses and antisera on an antigenic map so that distances between viruses and antisera predict observed serological data.
Here, we present a substantial advance to previous approaches that uses Bayesian multidimensional scaling (BMDS) to place antigenic cartography in a fully probabilistic framework.
Doing so has allowed us to build in important covariates into the statistical model; most importantly, phylogenetic relationships.
Thus, our method simultaneously models evolutionary and antigenic dynamics, and allows us to infer which phylogeny branches have evolved in antigenic phenotype and which branches have remained static.

We test our method with substantial new datasets for influenza A/H3N2, A/H1N1, B/Victoria and B/Yamagata, presenting, for the first time, detailed evolutionary and antigenic reconstructions for all four major circulating lineages.
We find that influenza A/H3N2 shows the fastest and most punctuated evolution of antigenic phenotype relative to the other three circulating lineages.
We connect antigenic evolution to incidence patterns, showing that antigenic drift leading into an influenza season accounts for 46\% of the variance in seasonal incidence.
This analysis shows a strong signal of interference between lineages, with antigenic drift in a sister lineage having a negative impact on within-lineage incidence.
Additionally, we've used phylogenetic information to characterize fundamental differences in antigenic dynamics between lineages, finding that A/H3N2 benefits from both a greater influx of new antigenic mutations and from a more rapid spread of these mutations through the population.

Both the link between incidence and antigenic drift and the presence of interference across influenza subtypes have remained long-standing questions in the influenza community.
We believe our results do much to address these questions.
Additionally, we believe our methodological advances will prove fundamental to future research in influenza and other antigenically variable pathogens.
Our unified genetic / antigenic approach will make it possible to look closely for causal relationships between genetic mutations and antigenic change, and to characterize, for the first time, the antigenic mutations that are ultimately successful in the global competition among virus lineages.

\end{newlfm}
\end{document}  